% integrals.tex
\section{Integrals}

In this section we will explore some types of exercises that can be asked in a typical calculus exam. Specifically we will explore two categories:

\begin{itemize}
	\item Integral function exercises
	\item Improper integral exercises
\end{itemize}

\subsection{Integral function exercises}

\begin{cascade}[Computing the derivative of a composition of the integral function]
	\textbf{Key Question:} How can we compute the derivative of a function that is the composition of the integral function with another function?
	\begin{itemize}
		\item \textbf{Remember the derivative formula for composite functions}
		      \begin{itemize}
			      \item $(f \circ g)'(x) = f'(g(x)) \cdot g'(x)$
		      \end{itemize}
		\item \textbf{Consider Torricelli's Theorem}
		      \begin{itemize}
			      \item $F(x) = \int_a^x f(t) dt$
			      \item $F'(x) = f(x)$
		      \end{itemize}
		\item \textbf{Example}
		      \begin{itemize}
			      \item $G(x) = \int_a^{m(x)} f(t)\,dt = F(m(x))$
			      \item $G'(x) = F'(m(x)) \cdot m'(x) = f(m(x)) \cdot m'(x)$
		      \end{itemize}
	\end{itemize}
\end{cascade}

\hfill

\begin{cascade}[Computing the derivative of an integral with variable limits]
	\textbf{Key Question:} How can we compute the derivative of a function defined by an integral where both limits depend on $x$, i.e., $G(x) = \int_{n(x)}^{m(x)} f(t) dt$?
	\begin{itemize}
		\item
		      \textbf{1. Split the integral using a constant lower limit}
		      \begin{itemize}
			      \item Use the property of definite integrals: $\int_a^b f(t) dt = \int_c^b f(t) dt - \int_c^a f(t) dt$.
			      \item Apply this to $G(x)$: Choose an arbitrary constant $a$ (often 0 or 1) in the domain of $f$.
			            \[ G(x) = \int_{n(x)}^{m(x)} f(t) dt = \int_a^{m(x)} f(t) dt - \int_a^{n(x)} f(t) dt \]
		      \end{itemize}

		\item
		      \textbf{2. Define an auxiliary function using Torricelli's Theorem}
		      \begin{itemize}
			      \item Let $F(x) = \int_a^x f(t) dt$.
			      \item By the Fundamental Theorem of Calculus (Part 1 / Torricelli's Theorem), $F'(x) = f(x)$.
			      \item Rewrite $G(x)$ using $F$: $G(x) = F(m(x)) - F(n(x))$.
		      \end{itemize}

		\item
		      \textbf{3. Differentiate using the Chain Rule}
		      \begin{itemize}
			      \item Apply the chain rule to differentiate $F(m(x))$ and $F(n(x))$:
			            \[ (F(h(x)))' = F'(h(x)) \cdot h'(x) = f(h(x)) \cdot h'(x) \]
			      \item Therefore, the derivative of $G(x)$ is:
			            \[ G'(x) = \frac{d}{dx}[F(m(x))] - \frac{d}{dx}[F(n(x))] \]
			            \[ G'(x) = F'(m(x)) \cdot m'(x) - F'(n(x)) \cdot n'(x) \]
			            \[ G'(x) = f(m(x)) \cdot m'(x) - f(n(x)) \cdot n'(x) \]
		      \end{itemize}

		\item
		      \textbf{4. Example}
		      \begin{itemize}
			      \item Compute the derivative of $G(x) = \int_x^{2x} \frac{\sin t}{t} dt$.
			      \item Here, $f(t) = \frac{\sin t}{t}$, $m(x) = 2x$, and $n(x) = x$.
			      \item Choose $a=1$ (any constant works). Let $F(x) = \int_1^x \frac{\sin t}{t} dt$.
			      \item Then $G(x) = F(2x) - F(x)$.
			      \item The derivatives are $m'(x) = 2$ and $n'(x) = 1$.
			      \item Applying the formula:
			            \[ G'(x) = f(m(x)) \cdot m'(x) - f(n(x)) \cdot n'(x) \]
			            \[ G'(x) = f(2x) \cdot 2 - f(x) \cdot 1 \]
			            \[ G'(x) = \frac{\sin(2x)}{2x} \cdot 2 - \frac{\sin x}{x} \cdot 1 \]
			            \[ G'(x) = \frac{\sin(2x)}{x} - \frac{\sin x}{x} = \frac{\sin(2x) - \sin x}{x} \]
		      \end{itemize}
	\end{itemize}
\end{cascade}

\hfill

\begin{cascade}[Computing limits involving integrals and indeterminate forms]
	\textbf{Key Question:} How to compute the limit: $$\lim_{x \to 0} \frac{x - \int_0^x (e^{-t^2} + \sin^2 t) dt}{x(x^2 - \sin^2 x)}$$
	\begin{itemize}
		\item \textbf{Initial Check: Identify Indeterminate Form}
		      \begin{itemize}
			      \item Substitute $x=0$ into the numerator: $0 - \int_0^0 (e^{-t^2} + \sin^2 t) dt = 0 - 0 = 0$.
			      \item Substitute $x=0$ into the denominator: $0 \cdot (0^2 - \sin^2 0) = 0 \cdot (0 - 0) = 0$.
			      \item The limit presents the indeterminate form $\left[\frac{0}{0}\right]$.
		      \end{itemize}
		\item \textbf{Apply L'Hôpital's Rule}
		      \begin{itemize}
			      \item Differentiate the numerator with respect to $x$. Requires the Fundamental Theorem of Calculus (Torricelli-Barrow) for the integral part:
			            \[ \frac{d}{dx} \left( x - \int_0^x (e^{-t^2} + \sin^2 t) dt \right) = 1 - (e^{-x^2} + \sin^2 x) \]
			      \item Differentiate the denominator with respect to $x$ using the product rule:
			            \[ \frac{d}{dx} (x(x^2 - \sin^2 x)) = (x^2 - \sin^2 x) + (2x^2 - x\sin(2x)) \]
			      \item The limit becomes (applying L'Hôpital's Rule):
			            \[ L \stackrel{H}{=} \lim_{x \to 0} \frac{1 - e^{-x^2} - \sin^2 x}{(x^2 - \sin^2 x) + (2x^2 - x\sin(2x))} \]
		      \end{itemize}
		\item \textbf{Evaluate the New Limit (Using Taylor Series Approximation)}
		      \begin{itemize}
			      \item The limit is still in the form $\left[\frac{0}{0}\right]$. Further application of L'Hôpital's Rule is possible, but Taylor expansions are often more efficient.
			      \item Approximate numerator and denominator for $x \to 0$:
			            \begin{itemize}
				            \item $1 - e^{-x^2} - \sin^2 x = \frac{5}{6}x^4 + o(x^4)$
				            \item $(x^2 - \sin^2 x) + (2x^2 - x\sin(2x)) = \frac{5}{3}x^4 + o(x^4)$
			            \end{itemize}
			      \item Substitute the approximations back into the limit:
			            \[ L = \lim_{x \to 0} \frac{\frac{5}{6}x^4 + o(x^4)}{\frac{5}{3}x^4 + o(x^4)} = \frac{5/6}{5/3} = \frac{5}{6} \cdot \frac{3}{5} = \frac{1}{2} \]
		      \end{itemize}
		\item \textbf{Result:} The limit exists and its value is $\frac{1}{2}$.
	\end{itemize}
\end{cascade}

\hfill

\begin{cascade}[Computing Maclaurin Polynomials for Composite Functions with Integrals]
	\textbf{Key Question:} How to determine the Maclaurin polynomial for $F(x) = \sin\left(\int_0^x e^{-t^2} dt\right)$?
	\begin{itemize}
		\item \textbf{Define Auxiliary Function for the Integral}
		      \begin{itemize}
			      \item Let $G(x) = \int_0^x e^{-t^2} dt$.
			      \item The original function becomes $F(x) = \sin(G(x))$.
		      \end{itemize}
		\item \textbf{Find the Maclaurin Series for the Auxiliary Function $G(x)$}
		      \begin{itemize}
			      \item Calculate derivatives of $G(x)$ and evaluate at $x=0$:
			            \begin{itemize}
				            \item $G(0) = \int_0^0 e^{-t^2} dt = 0$
				            \item $G'(x) = e^{-x^2}$ (by Fundamental Theorem of Calculus) $\rightarrow G'(0) = e^0 = 1$
				            \item $G''(x) = -2x e^{-x^2} \rightarrow G''(0) = 0$
				            \item $G'''(x) = -2e^{-x^2} + 4x^2 e^{-x^2} \rightarrow G'''(0) = -2$
			            \end{itemize}
			      \item Construct the Maclaurin series for $G(x)$ using $G(x) = \sum_{n=0}^{\infty} \frac{G^{(n)}(0)}{n!}x^n$:
			            \[ G(x) = G(0) + G'(0)x + \frac{G''(0)}{2!}x^2 + \frac{G'''(0)}{3!}x^3 + o(x^3) \]
			            \[ G(x) = 0 + 1 \cdot x + \frac{0}{2}x^2 + \frac{-2}{6}x^3 + o(x^3) = x - \frac{1}{3}x^3 + o(x^3) \]
		      \end{itemize}
		\item \textbf{Use the Known Maclaurin Series for the Outer Function ($\sin$)}
		      \begin{itemize}
			      \item Recall the Maclaurin series for $\sin(z)$:
			            \[ \sin(z) = z - \frac{z^3}{6} + o(z^3) \]
		      \end{itemize}
		\item \textbf{Substitute the Series for $G(x)$ into the Series for $\sin(z)$}
		      \begin{itemize}
			      \item Replace $z$ with $G(x) = x - \frac{1}{3}x^3 + o(x^3)$:
			            \[ F(x) = \sin(G(x)) = \left(x - \frac{1}{3}x^3 + o(x^3)\right) - \frac{1}{6}\left(x - \frac{1}{3}x^3 + o(x^3)\right)^3 + o\left((G(x))^3\right) \]
			      \item Since $G(x) \approx x$ for $x \to 0$, we have $o((G(x))^3) = o(x^3)$.
			      \item Expand and keep terms up to $x^3$:
			            \[ \left(x - \frac{1}{3}x^3 + o(x^3)\right)^3 = (x + o(x))^3 = x^3 + o(x^3) \]
			      \item Substitute back:
			            \[ F(x) = \left(x - \frac{1}{3}x^3\right) - \frac{1}{6}(x^3) + o(x^3) \]
			            \[ F(x) = x - \left(\frac{1}{3} + \frac{1}{6}\right)x^3 + o(x^3) = x - \frac{2+1}{6}x^3 + o(x^3) = x - \frac{3}{6}x^3 + o(x^3) \]
			            \[ F(x) = x - \frac{1}{2}x^3 + o(x^3) \]
		      \end{itemize}
		\item \textbf{Result: Maclaurin Polynomial}
		      \begin{itemize}
			      \item The Maclaurin polynomial of order 3 for $F(x)$ is $P_3(x) = x - \frac{1}{2}x^3$.
		      \end{itemize}
	\end{itemize}
\end{cascade}

\hfill

\begin{cascade}[Proving Existence and Uniqueness of Solutions Involving Integral Functions]
	\textbf{Key Question:} How to prove that the equation $f(x) = 1-x$, where $f(x) = \int_0^x e^{-t^2} dt$, has a unique solution?
	\begin{itemize}
		\item \textbf{Reformulate the Problem}
		      \begin{itemize}
			      \item Define a new function $g(x) = f(x) - (1-x) = f(x) + x - 1$.
			      \item The original problem is equivalent to proving that $g(x) = 0$ has exactly one solution (a unique zero).
		      \end{itemize}
		\item \textbf{Prove Existence of a Zero (Intermediate Value Theorem)}
		      \begin{itemize}
			      \item \textbf{Continuity:} $f(x)$ is continuous because it's an integral function of a continuous integrand ($e^{-t^2}$). The term $x-1$ is also continuous. Therefore, $g(x)$ is continuous on $\mathbb{R}$.
			      \item \textbf{Find points with opposite signs:}
			            \begin{itemize}
				            \item Evaluate $g(0)$: $g(0) = f(0) + 0 - 1 = \int_0^0 e^{-t^2} dt - 1 = 0 - 1 = -1$. So, $g(0) < 0$.
				            \item Evaluate the limit as $x \to +\infty$:
				                  \[ \lim_{x\to+\infty} g(x) = \lim_{x\to+\infty} \left( \int_0^x e^{-t^2} dt + x - 1 \right) \]
				                  The integral $\int_0^\infty e^{-t^2} dt$ converges to a finite value ($\frac{\sqrt{\pi}}{2}$). The term $x$ goes to $+\infty$.
				                  \[ \lim_{x\to+\infty} g(x) = \left( \lim_{x\to+\infty} \int_0^x e^{-t^2} dt \right) + \left( \lim_{x\to+\infty} x \right) - 1 = \frac{\sqrt{\pi}}{2} + \infty - 1 = +\infty \]
				            \item Since the limit is $+\infty$, there must exist some value $\bar{x}$ such that for all $x > \bar{x}$, $g(x) > 0$. Let's pick one such value $\bar{x}$.
			            \end{itemize}
			      \item \textbf{Apply IVT:} Since $g(x)$ is continuous on $[0, \bar{x}]$, $g(0) < 0$, and $g(\bar{x}) > 0$, the Intermediate Value Theorem guarantees that there exists at least one $x_0 \in (0, \bar{x})$ such that $g(x_0) = 0$.
		      \end{itemize}
		\item \textbf{Prove Uniqueness of the Zero (Monotonicity)}
		      \begin{itemize}
			      \item \textbf{Calculate the derivative $g'(x)$:}
			            \[ g'(x) = \frac{d}{dx} (f(x) + x - 1) = f'(x) + 1 \]
			      \item Apply the Fundamental Theorem of Calculus to find $f'(x)$:
			            \[ f'(x) = \frac{d}{dx} \left( \int_0^x e^{-t^2} dt \right) = e^{-x^2} \]
			      \item Substitute back into $g'(x)$:
			            \[ g'(x) = e^{-x^2} + 1 \]
			      \item \textbf{Analyze the sign of $g'(x)$:} Since $e^{-x^2} > 0$ for all real $x$, we have $g'(x) = e^{-x^2} + 1 > 0 + 1 = 1$.
			      \item Since $g'(x) > 0$ for all $x \in \mathbb{R}$, the function $g(x)$ is strictly increasing on its entire domain.
		      \end{itemize}
		\item \textbf{Conclusion}
		      \begin{itemize}
			      \item We have shown that $g(x)=0$ has at least one solution (by IVT).
			      \item We have shown that $g(x)$ is strictly increasing, which means it can cross the x-axis (equal zero) at most once.
			      \item Therefore, the equation $g(x)=0$ has exactly one unique solution. This implies the original equation $f(x)=1-x$ also has a unique solution.
		      \end{itemize}
	\end{itemize}
\end{cascade}

\hfill

\begin{cascade}[Finding Order and Principal Part for Functions with Integrals (Derivative Method)]
	\textbf{Key Question:} Determine the order and principal part of $G(x) = x - \int_0^x e^{-(t+x)^2} dt$ as $x \to 0$.
	\begin{itemize}
		\item \textbf{Strategy: Analyze the Derivative $G'(x)$}
		      \begin{itemize}
			      \item It's often easier to find the Maclaurin expansion (order/principal part) of the derivative first, and then integrate.
			      \item \textbf{Simplify the integral (Optional Variable Change):} Let $z=t+x$, $dz=dt$. Limits change from $t \in [0, x]$ to $z \in [x, 2x]$.
			            \[ G(x) = x - \int_x^{2x} e^{-z^2} dz \]
			      \item \textbf{Rewrite using a fixed lower limit:} Let $H(x) = \int_0^x e^{-t^2} dt$. Then $\int_x^{2x} e^{-z^2} dz = H(2x) - H(x)$.
			            \[ G(x) = x - (H(2x) - H(x)) = x - H(2x) + H(x) \]
		      \end{itemize}
		\item \textbf{Compute the Derivative $G'(x)$}
		      \begin{itemize}
			      \item Apply differentiation rules, including the Fundamental Theorem of Calculus and Chain Rule:
			            \[ G'(x) = \frac{d}{dx}(x) - \frac{d}{dx}(H(2x)) + \frac{d}{dx}(H(x)) \]
			            \[ G'(x) = 1 - (H'(2x) \cdot 2) + H'(x) \]
			            Since $H'(x) = e^{-x^2}$:
			            \[ G'(x) = 1 - (e^{-(2x)^2} \cdot 2) + e^{-x^2} = 1 - 2e^{-4x^2} + e^{-x^2} \]
		      \end{itemize}
		\item \textbf{Find Maclaurin Expansion of $G'(x)$}
		      \begin{itemize}
			      \item Use the known expansion $e^u = 1 + u + o(u)$ for $u \to 0$.
			      \item $e^{-4x^2} = 1 - 4x^2 + o(x^2)$
			      \item $e^{-x^2} = 1 - x^2 + o(x^2)$
			      \item Substitute into $G'(x)$:
			            \[ G'(x) = 1 - 2(1 - 4x^2 + o(x^2)) + (1 - x^2 + o(x^2)) \]
			            \[ G'(x) = 1 - 2 + 8x^2 + o(x^2) + 1 - x^2 + o(x^2) \]
			            \[ G'(x) = (1-2+1) + (8x^2 - x^2) + o(x^2) = 7x^2 + o(x^2) \]
			      \item The principal part of $G'(x)$ is $7x^2$, and its order is 2.
		      \end{itemize}
		\item \textbf{Determine Order and Principal Part of $G(x)$}
		      \begin{itemize}
			      \item Since $G'(x) \sim 7x^2$ as $x \to 0$, and $G(0) = 0 - \int_0^0 \dots = 0$, we can integrate the principal part of $G'(x)$ to find the principal part of $G(x)$:
			            \[ G(x) \approx \int_0^x 7t^2 dt = 7 \left[ \frac{t^3}{3} \right]_0^x = \frac{7}{3}x^3 \]
			      \item The principal part of $G(x)$ is $\frac{7}{3}x^3$.
			      \item The order of $G(x)$ is 3.
			      \item Thus, $G(x) = \frac{7}{3}x^3 + o(x^3)$ for $x \to 0$.
		      \end{itemize}
		\item \textbf{Verification (Using L'Hôpital's Rule on the Remainder)}
		      \begin{itemize}
			      \item To confirm $G(x) = \frac{7}{3}x^3 + o(x^3)$, we must show $\lim_{x \to 0} \frac{G(x) - \frac{7}{3}x^3}{x^3} = 0$.
			      \item Apply L'Hôpital's Rule (since form is $\frac{0}{0}$):
			            \[ \lim_{x \to 0} \frac{G'(x) - \frac{d}{dx}(\frac{7}{3}x^3)}{\frac{d}{dx}(x^3)} = \lim_{x \to 0} \frac{G'(x) - 7x^2}{3x^2} \]
			      \item Substitute the expansion of $G'(x)$:
			            \[ \lim_{x \to 0} \frac{(7x^2 + o(x^2)) - 7x^2}{3x^2} = \lim_{x \to 0} \frac{o(x^2)}{3x^2} = 0 \]
			      \item The limit is 0, confirming the order and principal part.
		      \end{itemize}
	\end{itemize}
\end{cascade}

\begin{cascade}[Qualitative Graph Sketching of Integral Functions near x=0]
	\textbf{Key Question:} How to draw a qualitative graph of $f(x) = \int_0^x \frac{e^t}{2t^2+1} dt$ in a neighborhood of $x=0$?
	\begin{itemize}
		\item \textbf{Strategy: Analyze Local Behavior using Derivatives at x=0}
		      \begin{itemize}
			      \item The behavior of a function near a point (like $x=0$) is determined by its value and the values of its derivatives at that point. This is the foundation of Taylor series approximations.
		      \end{itemize}
		\item \textbf{Calculate $f(0)$}
		      \begin{itemize}
			      \item Substitute $x=0$ into the integral definition:
			            \[ f(0) = \int_0^0 \frac{e^t}{2t^2+1} dt = 0 \]
			      \item The function passes through the origin $(0, 0)$.
		      \end{itemize}
		\item \textbf{Calculate $f'(x)$ and $f'(0)$}
		      \begin{itemize}
			      \item Apply the Fundamental Theorem of Calculus (Part 1):
			            \[ f'(x) = \frac{d}{dx} \left( \int_0^x \frac{e^t}{2t^2+1} dt \right) = \frac{e^x}{2x^2+1} \]
			      \item Evaluate at $x=0$:
			            \[ f'(0) = \frac{e^0}{2(0)^2+1} = \frac{1}{1} = 1 \]
			      \item The slope of the tangent line at the origin is $1$. The function is increasing at $x=0$.
		      \end{itemize}
		\item \textbf{Calculate $f''(x)$ and $f''(0)$}
		      \begin{itemize}
			      \item Differentiate $f'(x)$ using the quotient rule $\left(\frac{u}{v}\right)' = \frac{u'v - uv'}{v^2}$:
			            \[ u = e^x \implies u' = e^x \]
			            \[ v = 2x^2+1 \implies v' = 4x \]
			            \[ f''(x) = \frac{(e^x)(2x^2+1) - (e^x)(4x)}{(2x^2+1)^2} = \frac{e^x(2x^2 - 4x + 1)}{(2x^2+1)^2} \]
			      \item Evaluate at $x=0$:
			            \[ f''(0) = \frac{e^0(2(0)^2 - 4(0) + 1)}{(2(0)^2+1)^2} = \frac{1(1)}{(1)^2} = 1 \]
			      \item Since $f''(0) > 0$, the function is concave up at $x=0$.
		      \end{itemize}
		\item \textbf{Sketching the Graph near x=0}
		      \begin{itemize}
			      \item The graph passes through $(0, 0)$.
			      \item The tangent line at $(0, 0)$ has a slope of $1$ (like the line $y=x$).
			      \item The graph is concave up at $(0, 0)$, meaning it lies above its tangent line near the point of tangency.
			      \item Combining these: Start at the origin, draw a curve that is initially tangent to $y=x$ and curves upwards (concave up).
		      \end{itemize}
	\end{itemize}
\end{cascade}

\begin{cascade}[Limit Computation using Integral Inequalities and Squeeze Theorem]
	\textbf{Key Question:} Calculate the limit $L = \lim_{x \to 0^+} x \int_{x-x^2}^x \frac{dt}{\sin^3 t}$, if it exists.
	\begin{itemize}
		\item \textbf{Step 1: Analyze Integrand and Integration Interval}
		      \begin{itemize}
			      \item The integration variable is $t$, and the interval is $[x-x^2, x]$.
			      \item As $x \to 0^+$, both $x$ and $x-x^2 = x(1-x)$ approach $0^+$. Thus, $t \to 0^+$.
			      \item For $t \in (0, \pi/2)$, $\sin t$ is positive and strictly increasing.
			      \item Consequently, $\sin^3 t$ is also positive and strictly increasing for $t \in (0, \pi/2)$.
			      \item Therefore, the integrand $g(t) = \frac{1}{\sin^3 t}$ is positive and strictly decreasing for $t$ in the interval $(0, \pi/2)$.
		      \end{itemize}
		\item \textbf{Step 2: Establish Inequalities for the Integrand}
		      \begin{itemize}
			      \item Since $t \in [x-x^2, x]$ and $g(t) = \frac{1}{\sin^3 t}$ is decreasing on this interval (for sufficiently small positive $x$), the minimum value of $g(t)$ occurs at $t=x$ and the maximum value occurs at $t=x-x^2$.
			      \item For $t \in [x-x^2, x]$, we have:
			            \[ \frac{1}{\sin^3 x} \le \frac{1}{\sin^3 t} \le \frac{1}{\sin^3(x-x^2)} \]
		      \end{itemize}
		\item \textbf{Step 3: Integrate the Inequalities}
		      \begin{itemize}
			      \item Integrate all parts of the inequality over the interval $[x-x^2, x]$. Since the bounds $\frac{1}{\sin^3 x}$ and $\frac{1}{\sin^3(x-x^2)}$ are constant with respect to $t$, we get:
			            \[ \int_{x-x^2}^x \frac{1}{\sin^3 x} dt \le \int_{x-x^2}^x \frac{1}{\sin^3 t} dt \le \int_{x-x^2}^x \frac{1}{\sin^3(x-x^2)} dt \]
			      \item The length of the integration interval is $x - (x-x^2) = x^2$.
			            \[ \frac{1}{\sin^3 x} \cdot x^2 \le \int_{x-x^2}^x \frac{1}{\sin^3 t} dt \le \frac{1}{\sin^3(x-x^2)} \cdot x^2 \]
			            \[ \frac{x^2}{\sin^3 x} \le \int_{x-x^2}^x \frac{1}{\sin^3 t} dt \le \frac{x^2}{\sin^3(x-x^2)} \]
		      \end{itemize}
		\item \textbf{Step 4: Incorporate the External Factor and Compute Limits}
		      \begin{itemize}
			      \item Multiply the inequality by $x$ (note $x > 0$ as $x \to 0^+$):
			            \[ \frac{x^3}{\sin^3 x} \le x \int_{x-x^2}^x \frac{1}{\sin^3 t} dt \le \frac{x^3}{\sin^3(x-x^2)} \]
			      \item Compute the limit of the lower bound as $x \to 0^+$:
			            \[ \lim_{x \to 0^+} \frac{x^3}{\sin^3 x} = \lim_{x \to 0^+} \left(\frac{x}{\sin x}\right)^3 = (1)^3 = 1 \]
			            (using the standard limit $\lim_{u \to 0} \frac{\sin u}{u} = 1$)
			      \item Compute the limit of the upper bound as $x \to 0^+$:
			            \[ \lim_{x \to 0^+} \frac{x^3}{\sin^3(x-x^2)} = \lim_{x \to 0^+} \left(\frac{x}{\sin(x-x^2)}\right)^3 \]
			            We use $\sin(x-x^2) \sim x-x^2$ as $x \to 0$.
			            \[ \lim_{x \to 0^+} \left(\frac{x}{x-x^2}\right)^3 = \lim_{x \to 0^+} \left(\frac{x}{x(1-x)}\right)^3 = \lim_{x \to 0^+} \left(\frac{1}{1-x}\right)^3 = \left(\frac{1}{1-0}\right)^3 = 1 \]
		      \end{itemize}
		\item \textbf{Step 5: Apply the Squeeze Theorem}
		      \begin{itemize}
			      \item We have shown:
			            \[ \lim_{x \to 0^+} \frac{x^3}{\sin^3 x} = 1 \]
			            \[ \lim_{x \to 0^+} \frac{x^3}{\sin^3(x-x^2)} = 1 \]
			            \[ \frac{x^3}{\sin^3 x} \le x \int_{x-x^2}^x \frac{1}{\sin^3 t} dt \le \frac{x^3}{\sin^3(x-x^2)} \]
			      \item By the Squeeze Theorem, the limit of the middle expression must also be 1.
			      \item Therefore, $L = \lim_{x \to 0^+} x \int_{x-x^2}^x \frac{dt}{\sin^3 t} = 1$.
		      \end{itemize}
	\end{itemize}
\end{cascade}

\hfill

\begin{cascade}[Limit Computation using Mean Value Theorem and Squeeze Theorem]
	\textbf{Key Question:} Calculate the limit $L = \lim_{x \to +\infty} x^3 \int_{x^2}^{x^2+x} \sin\left(\frac{1}{t^2}\right) dt$.
	\begin{itemize}
		\item \textbf{Step 1: Apply Mean Value Theorem for Integrals}
		      \begin{itemize}
			      \item \textbf{Theorem Statement:} If $f$ is continuous on $[a, b]$, there exists $c \in [a, b]$ such that $\int_a^b f(t) dt = f(c)(b-a)$.
			      \item \textbf{Application:}
			            \begin{itemize}
				            \item Let $f(t) = \sin\left(\frac{1}{t^2}\right)$. This function is continuous on $[x^2, x^2+x]$ for large $x$ (since $x^2 > 0$).
				            \item The interval length is $(x^2+x) - x^2 = x$.
				            \item By the MVT for Integrals, there exists $z \in [x^2, x^2+x]$ such that:
				                  \[ \int_{x^2}^{x^2+x} \sin\left(\frac{1}{t^2}\right) dt = \sin\left(\frac{1}{z^2}\right) \cdot x \]
			            \end{itemize}
			      \item \textbf{Rewrite the Limit:} Substitute this result back into the limit expression:
			            \[ L = \lim_{x \to +\infty} x^3 \left[ x \sin\left(\frac{1}{z^2}\right) \right] = \lim_{x \to +\infty} x^4 \sin\left(\frac{1}{z^2}\right) \]
			            where $z$ depends on $x$ and satisfies $x^2 \le z \le x^2+x$.
		      \end{itemize}
		\item \textbf{Step 2: Establish Bounds using $z \in [x^2, x^2+x]$}
		      \begin{itemize}
			      \item From $x^2 \le z \le x^2+x$, we have:
			            \[ \frac{1}{x^2+x} \le \frac{1}{z} \le \frac{1}{x^2} \]
			            Squaring (all terms are positive for large $x$):
			            \[ \frac{1}{(x^2+x)^2} \le \frac{1}{z^2} \le \frac{1}{x^4} \]
			      \item As $x \to +\infty$, all terms in the inequality approach $0^+$.
			      \item Since $\sin u$ is an increasing function for $u$ near $0^+$, we can apply $\sin$ to the inequalities (for sufficiently large $x$):
			            \[ \sin\left(\frac{1}{(x^2+x)^2}\right) \le \sin\left(\frac{1}{z^2}\right) \le \sin\left(\frac{1}{x^4}\right) \]
		      \end{itemize}
		\item \textbf{Step 3: Apply the Squeeze Theorem}
		      \begin{itemize}
			      \item Multiply the inequality by $x^4$ (which is positive):
			            \[ x^4 \sin\left(\frac{1}{(x^2+x)^2}\right) \le x^4 \sin\left(\frac{1}{z^2}\right) \le x^4 \sin\left(\frac{1}{x^4}\right) \]
			      \item \textbf{Limit of the Upper Bound:}
			            \[ \lim_{x \to +\infty} x^4 \sin\left(\frac{1}{x^4}\right) \]
			            Use the standard limit $\lim_{u \to 0} \frac{\sin u}{u} = 1$. Let $u = 1/x^4$. As $x \to +\infty$, $u \to 0^+$.
			            \[ = \lim_{u \to 0^+} \frac{1}{u} \sin(u) = 1 \]
			      \item \textbf{Limit of the Lower Bound:}

			            Let $v = 1/(x^2+x)^2$. As $x \to +\infty$, $v \to 0^+$. Use $\sin v \sim v$ for $v \to 0$.
			            \[ \lim_{x \to +\infty} x^4 \sin\left(\frac{1}{(x^2+x)^2}\right) \]
			            \[ = \lim_{x \to +\infty} x^4 \cdot \frac{1}{(x^2+x)^2} = \lim_{x \to +\infty} \frac{x^4}{(x^2(1+1/x))^2} = \lim_{x \to +\infty} \frac{x^4}{x^4(1+1/x)^2} = 1 \]
			      \item \textbf{Conclusion:} Since the expression $x^4 \sin(1/z^2)$ is squeezed between two functions that both tend to 1 as $x \to +\infty$, by the Squeeze Theorem (Teorema dei Carabinieri):
			            \[ L = \lim_{x \to +\infty} x^4 \sin\left(\frac{1}{z^2}\right) = 1 \]
		      \end{itemize}
	\end{itemize}
\end{cascade}

\hfill

\begin{cascade}[Integration using Weierstrass Substitution (Tangent Half-Angle)]
	\textbf{Key Question:} How to calculate the integral $I = \int \frac{\tan x}{\sin x + \tan x} dx$?
	\begin{itemize}
		\item \textbf{Step 1: Simplify the Integrand}
		      \begin{itemize}
			      \item Rewrite $\tan x$ as $\sin x / \cos x$:
			            \[ \frac{\tan x}{\sin x + \tan x} = \frac{\frac{\sin x}{\cos x}}{\sin x + \frac{\sin x}{\cos x}} \]
			      \item Find a common denominator in the denominator:
			            \[ \frac{\frac{\sin x}{\cos x}}{\sin x + \frac{\sin x}{\cos x}} = \frac{\frac{\sin x}{\cos x}}{\frac{\sin x \cos x + \sin x}{\cos x}} \]
			      \item Simplify (assuming $\sin x \neq 0, \cos x \neq 0$):
			            \[ \frac{\frac{\sin x}{\cos x}}{\frac{\sin x \cos x + \sin x}{\cos x}} = \frac{\sin x}{\sin x \cos x + \sin x} = \frac{\sin x}{\sin x (\cos x + 1)} = \frac{1}{\cos x + 1} \]
			      \item The integral becomes significantly simpler: $I = \int \frac{1}{\cos x + 1} dx$.
		      \end{itemize}
		\item \textbf{Step 2: Apply the Weierstrass Substitution ($t = \tan(x/2)$)}
		      \begin{itemize}
			      \item \textbf{Motivation:} This substitution transforms rational functions of $\sin x$ and $\cos x$ into rational functions of $t$, which can often be integrated using standard techniques (like partial fractions, although not needed here).
			      \item \textbf{The Substitution Formulas:}
			            \begin{itemize}
				            \item Let $\boxed{t = \tan\left(\frac{x}{2}\right)}$
				            \item Then $\cos x = \frac{1-t^2}{1+t^2}$
				            \item And $dx = \frac{2}{1+t^2} dt$
				            \item (Also $\sin x = \frac{2t}{1+t^2}$, though not required for this specific simplified integral).
			            \end{itemize}
		      \end{itemize}
		\item \textbf{Step 3: Substitute into the Simplified Integral}
		      \begin{itemize}
			      \item Replace $\cos x$ and $dx$ with their $t$-equivalents:
			            \[ I = \int \underbrace{\frac{1}{\left(\frac{1-t^2}{1+t^2}\right) + 1}}_{\frac{1}{\cos x + 1}} \cdot \underbrace{\left(\frac{2}{1+t^2}\right) dt}_{dx} \]
			      \item Simplify the first part (the integrand in terms of $x$ expressed in $t$):
			            \[ \frac{1}{\frac{1-t^2 + 1+t^2}{1+t^2}} = \frac{1}{\frac{2}{1+t^2}} = \frac{1+t^2}{2} \]
			      \item Substitute this back into the integral expression:
			            \[ I = \int \left(\frac{1+t^2}{2}\right) \cdot \left(\frac{2}{1+t^2}\right) dt \]
		      \end{itemize}
		\item \textbf{Step 4: Evaluate the Integral in $t$}
		      \begin{itemize}
			      \item The integrand simplifies dramatically:
			            \[ I = \int 1 \, dt \]
			      \item Compute the straightforward integral:
			            \[ I = t + C \]
		      \end{itemize}
		\item \textbf{Step 5: Substitute Back to $x$}
		      \begin{itemize}
			      \item Replace $t$ with its definition in terms of $x$:
			            \[ I = \tan\left(\frac{x}{2}\right) + C \]
		      \end{itemize}
		\item \textbf{Result:} The integral evaluates to $\int \frac{\tan x}{\sin x + \tan x} dx = \tan\left(\frac{x}{2}\right) + C$.
	\end{itemize}
\end{cascade}

\clearpage

\subsection{Improper Integral exercises}

\begin{cascade}[Evaluating Improper Integrals using the Definition]
	\textbf{Key Question:} How to compute the value of the improper integral $\int_0^1 \log x \, dx$?
	\begin{itemize}
		\item \textbf{Step 1: Identify the Point of Impropriety}
		      \begin{itemize}
			      \item The function $f(x) = \log x$ is defined on $(0, 1]$.
			      \item The integral is improper because the integrand $\log x$ approaches $-\infty$ as $x$ approaches the lower limit of integration, $x=0$.
		      \end{itemize}
		\item \textbf{Step 2: Apply the Definition of Improper Integral}
		      \begin{itemize}
			      \item Replace the problematic lower limit $0$ with a variable $\epsilon$ and take the limit as $\epsilon$ approaches $0$ from the right side (since the interval is $(0, 1]$):
			            \[ \int_0^1 \log x \, dx \overset{\text{def}}{=} \lim_{\epsilon \to 0^+} \int_\epsilon^1 \log x \, dx \]
		      \end{itemize}
		\item \textbf{Step 3: Find the Antiderivative}
		      \begin{itemize}
			      \item Compute the indefinite integral $\int \log x \, dx$ using integration by parts:
			            \begin{itemize}
				            \item Let $u = \log x \implies du = \frac{1}{x} dx$
				            \item Let $dv = dx \implies v = x$
			            \end{itemize}
			            \[ \int \log x \, dx = uv - \int v \, du = x \log x - \int x \left(\frac{1}{x}\right) dx \]
			            \[ = x \log x - \int 1 \, dx = x \log x - x + C = x(\log x - 1) + C \]
			      \item An antiderivative is $F(x) = x(\log x - 1)$.
		      \end{itemize}
		\item \textbf{Step 4: Evaluate the Definite Integral and Compute the Limit}
		      \begin{itemize}
			      \item Use the antiderivative to evaluate the definite integral inside the limit (Fundamental Theorem of Calculus / Torricelli-Barrow):
			            \[ \int_\epsilon^1 \log x \, dx = [x(\log x - 1)]_\epsilon^1 = (1(\log 1 - 1)) - (\epsilon(\log \epsilon - 1)) \]
			            \[ = (1(0 - 1)) - (\epsilon \log \epsilon - \epsilon) = -1 - \epsilon \log \epsilon + \epsilon \]
			      \item Now, compute the limit as $\epsilon \to 0^+$:
			            \[ \lim_{\epsilon \to 0^+} (-1 - \epsilon \log \epsilon + \epsilon) \]
			      \item Use the standard limit $\lim_{\epsilon \to 0^+} \epsilon \log \epsilon = 0$.
			            \[ = -1 - 0 + 0 = -1 \]
		      \end{itemize}
		\item \textbf{Conclusion}
		      \begin{itemize}
			      \item Since the limit exists and is finite, the improper integral converges.
			      \item The value of the integral is $\int_0^1 \log x \, dx = -1$.
		      \end{itemize}
	\end{itemize}
\end{cascade}

\hfill

\begin{cascade}[Integration using Weierstrass Substitution (Tangent Half-Angle)]
	\textbf{Key Question:} How to calculate the integral $I = \int \frac{\tan x}{\sin x + \tan x} dx$?
	\begin{itemize}
		\item \textbf{Step 1: Simplify the Integrand}
		      \begin{itemize}
			      \item Rewrite $\tan x$ as $\sin x / \cos x$:
			            \[ \frac{\tan x}{\sin x + \tan x} = \frac{\frac{\sin x}{\cos x}}{\sin x + \frac{\sin x}{\cos x}} \]
			      \item Find a common denominator in the denominator:
			            \[ = \frac{\frac{\sin x}{\cos x}}{\frac{\sin x \cos x + \sin x}{\cos x}} \]
			      \item Simplify (assuming $\sin x \neq 0, \cos x \neq 0$):
			            \[ = \frac{\sin x}{\sin x \cos x + \sin x} = \frac{\sin x}{\sin x (\cos x + 1)} = \frac{1}{\cos x + 1} \]
			      \item The integral becomes significantly simpler: $I = \int \frac{1}{\cos x + 1} dx$.
		      \end{itemize}
		\item \textbf{Step 2: Apply the Weierstrass Substitution ($t = \tan(x/2)$)}
		      \begin{itemize}
			      \item \textbf{Motivation:} This substitution transforms rational functions of $\sin x$ and $\cos x$ into rational functions of $t$, which can often be integrated using standard techniques (like partial fractions, although not needed here).
			      \item \textbf{The Substitution Formulas:}
			            \begin{itemize}
				            \item Let $\boxed{t = \tan\left(\frac{x}{2}\right)}$
				            \item Then $\cos x = \frac{1-t^2}{1+t^2}$
				            \item And $dx = \frac{2}{1+t^2} dt$
				            \item (Also $\sin x = \frac{2t}{1+t^2}$, though not required for this specific simplified integral).
			            \end{itemize}
		      \end{itemize}
		\item \textbf{Step 3: Substitute into the Simplified Integral}
		      \begin{itemize}
			      \item Replace $\cos x$ and $dx$ with their $t$-equivalents:
			            \[ I = \int \underbrace{\frac{1}{\left(\frac{1-t^2}{1+t^2}\right) + 1}}_{\frac{1}{\cos x + 1}} \cdot \underbrace{\left(\frac{2}{1+t^2}\right) dt}_{dx} \]
			      \item Simplify the first part (the integrand in terms of $x$ expressed in $t$):
			            \[ \frac{1}{\frac{1-t^2 + 1+t^2}{1+t^2}} = \frac{1}{\frac{2}{1+t^2}} = \frac{1+t^2}{2} \]
			      \item Substitute this back into the integral expression:
			            \[ I = \int \left(\frac{1+t^2}{2}\right) \cdot \left(\frac{2}{1+t^2}\right) dt \]
		      \end{itemize}
		\item \textbf{Step 4: Evaluate the Integral in $t$}
		      \begin{itemize}
			      \item The integrand simplifies dramatically:
			            \[ I = \int 1 \, dt \]
			      \item Compute the straightforward integral:
			            \[ I = t + C \]
		      \end{itemize}
		\item \textbf{Step 5: Substitute Back to $x$}
		      \begin{itemize}
			      \item Replace $t$ with its definition in terms of $x$:
			            \[ I = \tan\left(\frac{x}{2}\right) + C \]
		      \end{itemize}
		\item \textbf{Result:} The integral evaluates to $\int \frac{\tan x}{\sin x + \tan x} dx = \tan\left(\frac{x}{2}\right) + C$.
	\end{itemize}
\end{cascade}

\hfill

\begin{cascade}[Convergence of Improper Integrals with Oscillating Numerators]
	\textbf{Key Question:} How to discuss the convergence of the improper integral $I = \int_0^\infty \frac{\cos x}{1+x^2} dx$?
	\begin{itemize}
		\item \textbf{Step 1: Identify Potential Issues}
		      \begin{itemize}
			      \item \textbf{Infinite Interval:} The upper limit is $+\infty$, making it an improper integral of the first kind.
			      \item \textbf{Sign Changes:} The numerator $\cos x$ oscillates between -1 and 1, causing the integrand $f(x) = \frac{\cos x}{1+x^2}$ to change sign infinitely often.
			      \item \textbf{Problem with Direct Comparison:} Standard comparison tests (Direct Comparison, Limit Comparison) require the integrand to be non-negative (at least eventually), which is not the case here.
		      \end{itemize}
		\item \textbf{Step 2: Use Absolute Convergence}
		      \begin{itemize}
			      \item \textbf{Strategy:} Test the convergence of the integral of the absolute value of the integrand:
			            \[ I_{abs} = \int_0^\infty \left| \frac{\cos x}{1+x^2} \right| dx = \int_0^\infty \frac{|\cos x|}{1+x^2} dx \]
			      \item \textbf{Absolute Convergence Theorem:} If $I_{abs}$ converges, then the original integral $I$ also converges.
			      \item The new integrand $\frac{|\cos x|}{1+x^2}$ is now non-negative, allowing the use of comparison tests.
		      \end{itemize}
		\item \textbf{Step 3: Apply Direct Comparison Test to $I_{abs}$}
		      \begin{itemize}
			      \item \textbf{Establish Inequality:} Use the property $|\cos x| \le 1$ for all $x$:
			            \[ 0 \le \frac{|\cos x|}{1+x^2} \le \frac{1}{1+x^2} \quad \text{for all } x \ge 0 \]
			      \item \textbf{Analyze the Comparison Integral:} Consider the integral $J = \int_0^\infty \frac{1}{1+x^2} dx$.
			      \item \textbf{Convergence of $J$ (Method 1: Direct Integration):}
			            \[ J = \int_0^\infty \frac{1}{1+x^2} dx = [\arctan x]_0^\infty = \lim_{b \to \infty} (\arctan b) - \arctan 0 = \frac{\pi}{2} - 0 = \frac{\pi}{2} \]
			            Since the result is finite, $J$ converges.
			      \item \textbf{Convergence of $J$ (Method 2: Splitting \& Asymptotic Comparison - Important General Technique):}
			            \begin{itemize}
				            \item Split the integral: $J = \int_0^1 \frac{1}{1+x^2} dx + \int_1^\infty \frac{1}{1+x^2} dx$.
				            \item The integral $\int_0^1 \frac{1}{1+x^2} dx$ is proper and finite (integrand is continuous on $[0,1]$).
				            \item For $\int_1^\infty \frac{1}{1+x^2} dx$, compare with $\int_1^\infty \frac{1}{x^2} dx$. Since $0 < \frac{1}{1+x^2} \le \frac{1}{x^2}$ for $x \ge 1$ and $\int_1^\infty \frac{1}{x^2} dx$ converges (p-integral with $p=2>1$), the integral $\int_1^\infty \frac{1}{1+x^2} dx$ converges by Direct Comparison.
				            \item Since both parts converge, $J$ converges. (Note: Splitting is crucial when dealing with multiple improprieties or when asymptotic comparison is needed at different points).
			            \end{itemize}
			      \item \textbf{Conclusion of Comparison Test:} Since $0 \le \frac{|\cos x|}{1+x^2} \le \frac{1}{1+x^2}$ and $\int_0^\infty \frac{1}{1+x^2} dx$ converges, by the Direct Comparison Test, $I_{abs} = \int_0^\infty \frac{|\cos x|}{1+x^2} dx$ must also converge.
		      \end{itemize}
		\item \textbf{Step 4: Final Conclusion}
		      \begin{itemize}
			      \item We have shown that $I_{abs} = \int_0^\infty \left| \frac{\cos x}{1+x^2} \right| dx$ converges.
			      \item By the theorem on absolute convergence, if an integral converges absolutely, it also converges simply.
			      \item Therefore, the original integral $I = \int_0^\infty \frac{\cos x}{1+x^2} dx$ converges.
		      \end{itemize}
	\end{itemize}
\end{cascade}

\hfill

\begin{cascade}[Determining Improper Integral Convergence using Asymptotic Comparison]
	\textbf{Key Question:} How can we determine if an improper integral converges when the integrand is complex, especially near a point of discontinuity?

	\begin{itemize}
		\item \textbf{Problem Setup:} Consider the integral:
		      $$ I = \int_0^1 f(x) dx = \int_0^1 \frac{\sqrt{x(x^2+1)} e^{2x}}{\tan \sqrt[3]{x}} dx $$
		      The potential problem (discontinuity) is at $x=0$.

		\item \textbf{Check Conditions for Comparison Test:}
		      \begin{itemize}
			      \item The function $f(x)$ must be continuous on $(0, 1]$.
			      \item The function $f(x)$ must be non-negative on $(0, 1]$.
			      \item In this case, $f(x)$ is non-negative for $x \in (0, 1]$.
		      \end{itemize}

		\item \textbf{Apply Asymptotic Comparison Test (Limit Comparison Test):}
		      We need to find a simpler function $g(x)$ such that $f(x) \sim g(x)$ as $x \to 0^+$. The convergence of $\int_0^1 f(x) dx$ will be the same as the convergence of $\int_0^1 g(x) dx$.
		      \begin{itemize}
			      \item \textbf{Analyze numerator as $x \to 0^+$:}
			            $$ \sqrt{x(x^2+1)} = \sqrt{x^3+x} \sim \sqrt{x} = x^{1/2} $$
			            $$ e^{2x} \sim e^0 = 1 $$
			            So, Numerator $\sim x^{1/2}$.
			      \item \textbf{Analyze denominator as $x \to 0^+$:}
			            $$ \tan \sqrt{x} \sim \sqrt{x} = x^{1/2} $$
			            $$ \tan \sqrt[3]{x} \sim x^{1/3} = x^{1/3} $$
			      \item \textbf{Determine asymptotic behavior of $f(x)$:}
			            $$ f(x) \sim \frac{\sqrt{x}}{\sqrt[3]{x}} = \frac{x^{1/2}}{x^{1/3}} = x^{1/2 - 1/3} = x^{1/6} $$
			            So, we use $g(x) = x^{1/6}$.
		      \end{itemize}

		\item \textbf{Analyze the Comparison Integral:}
		      We examine the convergence of $\int_0^1 g(x) dx$:
		      $$ \int_0^1 x^{1/6} dx = \int_0^1 \frac{1}{x^{-1/6}} dx $$
		      This is a generalized p-integral $\int_0^1 \frac{1}{x^p} dx$. It converges if and only if $p < 1$.
		      In our case, $p = -1/6$. Since $p = -1/6 < 1$, the comparison integral $\int_0^1 x^{1/6} dx$ converges.



		\item \textbf{Conclusion:}
		      Since $f(x) \sim x^{1/6}$ as $x \to 0^+$ and $\int_0^1 x^{1/6} dx$ converges, by the Limit Comparison Test, the original integral
		      $$ \int_0^1 \frac{\sqrt{x(x^2+1)} e^{2x}}{\tan \sqrt[3]{x}} dx \quad \textbf{converges.} $$
	\end{itemize}
\end{cascade}

\hfill

\begin{cascade}[Convergence of Improper Integrals with Parameters using Taylor Expansion]
	\textbf{Key Question:} How can we determine the values of a parameter $\alpha$ for which an improper integral converges, especially when the integrand's behavior near the problematic point requires Taylor series?

	\begin{itemize}
		\item \textbf{Problem Setup:} Consider the integral dependent on $\alpha$:
		      $$ I(\alpha) = \int_0^1 \frac{3}{[2(x - \log(1+x))]^{3-3\alpha}} dx $$
		      We need to find the values of $\alpha$ for which this integral converges. The potential problem point is $x=0$.

		\item \textbf{Asymptotic Analysis near $x=0$ using Taylor Expansion:}
		      To understand the behavior of the integrand $f(x, \alpha) = \frac{3}{[2(x - \log(1+x))]^{3-3\alpha}}$ as $x \to 0^+$, we need the Taylor expansion of $\log(1+x)$ around $x=0$.
		      \begin{itemize}
			      \item \textbf{Recall Taylor expansion:}
			            $$ \log(1+x) = x - \frac{x^2}{2} + \frac{x^3}{3} - \dots = x - \frac{x^2}{2} + o(x^2) \quad \text{as } x \to 0 $$
			      \item \textbf{Analyze the term $x - \log(1+x)$:}
			            $$ x - \log(1+x) = x - \left( x - \frac{x^2}{2} + o(x^2) \right) = \frac{x^2}{2} + o(x^2) $$
			            Therefore, $x - \log(1+x) \sim \frac{x^2}{2}$ as $x \to 0^+$.
			      \item \textbf{Analyze the base of the power in the denominator:}
			            $$ 2(x - \log(1+x)) \sim 2 \left( \frac{x^2}{2} \right) = x^2 \quad \text{as } x \to 0^+ $$
			      \item \textbf{Determine the asymptotic behavior of the integrand $f(x, \alpha)$:}
			            $$ f(x, \alpha) = \frac{3}{[2(x - \log(1+x))]^{3-3\alpha}} \sim \frac{3}{[x^2]^{3-3\alpha}} = \frac{3}{x^{6-6\alpha}} \quad \text{as } x \to 0^+ $$
		      \end{itemize}

		\item \textbf{Apply Asymptotic Comparison Test:}
		      The convergence of $I(\alpha)$ is equivalent to the convergence of the comparison integral:
		      $$ \int_0^1 \frac{3}{x^{6-6\alpha}} dx = 3 \int_0^1 \frac{1}{x^{6-6\alpha}} dx $$

		\item \textbf{Analyze the Comparison Integral (Generalized p-integral):}
		      The integral $\int_0^1 \frac{1}{x^p} dx$ converges if and only if $p < 1$.
		      In our case, the exponent is $p = 6 - 6\alpha$.

		\item \textbf{Determine the Condition for Convergence:}
		      For convergence, we require $p < 1$:
		      $$ 6 - 6\alpha < 1 $$
		      $$ 5 < 6\alpha $$
		      $$ \alpha > \frac{5}{6} $$

		\item \textbf{Conclusion:}
		      Based on the asymptotic comparison test and the condition for p-integral convergence, the original integral
		      $$ \int_0^1 \frac{3}{[2(x - \log(1+x))]^{3-3\alpha}} dx \quad \textbf{converges if and only if } \alpha > \frac{5}{6}. $$
	\end{itemize}
\end{cascade}

\clearpage

% Add further cascade schemes about integrals here
