\section{Multivariate Calculus}

In this section we will explore different types of exercises about multivariate calculus.

\subsection{Limits exercises}

In this subsection, we will tackle exercises about limits of multivariate functions and explore the different methods we can use for solving them or at least,
prove that they exist or not.

\begin{cascade}[Studying continuity of a two-variable function at a point]
	\textbf{Key Question:} How to determine if a piecewise function $f(x, y)$ is continuous at a point where its definition changes, typically $(0,0)$?

	\textbf{Function Example:}
	\[ f(x, y) = \begin{cases} \frac{x^2 y^2}{x^2 + y^2} & \text{if } (x, y) \neq (0, 0) \\ 0 & \text{if } (x, y) = (0, 0) \end{cases} \]
	\begin{itemize}
		\item \textbf{Check continuity away from the critical point:}
		      \begin{itemize}
			      \item For $(x, y) \neq (0, 0)$, $f(x, y) = \frac{x^2 y^2}{x^2 + y^2}$. This is a ratio of polynomials (which are continuous everywhere).
			      \item The denominator $x^2 + y^2$ is zero only at $(0, 0)$.
			      \item Therefore, $f(x, y)$ is continuous for all $(x, y) \neq (0, 0)$ as it's a composition/ratio of continuous functions. The domain is $\mathbb{R}^2$.
		      \end{itemize}
		\item \textbf{Check continuity at the critical point $(0, 0)$:}
		      \begin{itemize}
			      \item \textbf{Definition of continuity:} We need to verify if $\lim_{(x,y) \to (0,0)} f(x, y) = f(0, 0)$.
			      \item We are given $f(0, 0) = 0$.
			      \item We need to compute $\lim_{(x,y) \to (0,0)} \frac{x^2 y^2}{x^2 + y^2}$. If the limit exists and equals 0, the function is continuous at $(0,0)$.
		      \end{itemize}
		\item \textbf{Method 1: Inequality / Squeeze Theorem}
		      \begin{itemize}
			      \item Find bounds for the expression. Observe that $y^2 \le x^2 + y^2$.
			      \item For $(x, y) \neq (0, 0)$, divide by $x^2+y^2$: $\frac{y^2}{x^2 + y^2} \le 1$.
			      \item Since $x^2 \ge 0$, we can multiply by $x^2$: $x^2 \frac{y^2}{x^2 + y^2} \le x^2$.
			      \item The full expression is non-negative: $0 \le \frac{x^2 y^2}{x^2 + y^2} \le x^2$.
			      \item Take the limit as $(x,y) \to (0,0)$:
			            \[ 0 \le \lim_{(x,y) \to (0,0)} \frac{x^2 y^2}{x^2 + y^2} \le \lim_{(x,y) \to (0,0)} x^2 \]
			      \item Since $\lim_{(x,y) \to (0,0)} 0 = 0$ and $\lim_{(x,y) \to (0,0)} x^2 = 0$.
			      \item By the Squeeze Theorem, $\lim_{(x,y) \to (0,0)} \frac{x^2 y^2}{x^2 + y^2} = 0$.
			      \item \textit{Note:} If the expression involved terms that could be negative, absolute values would be necessary (e.g., $|f(x,y)| \le g(x,y)$ where $g \to 0$).
		      \end{itemize}
		\item \textbf{Method 2: Polar Coordinates}
		      \begin{itemize}
			      \item Substitute $x = \rho \cos \theta$ and $y = \rho \sin \theta$. The limit $(x, y) \to (0, 0)$ corresponds to $\rho \to 0^+$.
			      \item Substitute into the expression:
			            \[ \frac{(\rho \cos \theta)^2 (\rho \sin \theta)^2}{(\rho \cos \theta)^2 + (\rho \sin \theta)^2} = \frac{\rho^4 \cos^2 \theta \sin^2 \theta}{\rho^2 (\cos^2 \theta + \sin^2 \theta)} = \frac{\rho^4 \cos^2 \theta \sin^2 \theta}{\rho^2} = \rho^2 \cos^2 \theta \sin^2 \theta \]
			      \item Compute the limit as $\rho \to 0^+$: $\lim_{\rho \to 0^+} (\rho^2 \cos^2 \theta \sin^2 \theta)$.
			      \item Use bounds for the trigonometric part: $0 \le \cos^2 \theta \sin^2 \theta \le 1$. The value depends on $\theta$ but is bounded.
			      \item Multiply by $\rho^2$: $0 \le \rho^2 \cos^2 \theta \sin^2 \theta \le \rho^2$.
			      \item Apply the Squeeze Theorem for the limit in $\rho$: Since $\lim_{\rho \to 0^+} 0 = 0$ and $\lim_{\rho \to 0^+} \rho^2 = 0$.
			      \item Therefore, $\lim_{\rho \to 0^+} (\rho^2 \cos^2 \theta \sin^2 \theta) = 0$. The limit exists and is independent of $\theta$.
		      \end{itemize}
		\item \textbf{Conclusion for $(0, 0)$:} Both methods show $\lim_{(x,y) \to (0,0)} f(x, y) = 0$. Since this equals $f(0, 0)$, the function is continuous at $(0, 0)$.
		\item \textbf{Overall Conclusion:} Since $f(x, y)$ is continuous for $(x,y) \neq (0,0)$ and also continuous at $(0,0)$, the function is continuous on its entire domain $\mathbb{R}^2$.
	\end{itemize}
\end{cascade}

\hfill

\begin{cascade}[Proving discontinuity using path restriction]
	\textbf{Key Question:} How can we show a function $f(x, y)$ is \textit{not} continuous at a point, for example, $(0,0)$?

	\textbf{Function Example:}
	\[ f(x, y) = \begin{cases} \arctan\left(\frac{x}{x^2 + y^2}\right) & \text{if } (x, y) \neq (0, 0) \\ \frac{\pi}{2} & \text{if } (x, y) = (0, 0) \end{cases} \]
	\textbf{Domain:} $\mathbb{R}^2$.
	\begin{itemize}
		\item \textbf{Check continuity away from the critical point:}
		      \begin{itemize}
			      \item For $(x, y) \neq (0, 0)$, $f(x, y) = \arctan\left(\frac{x}{x^2 + y^2}\right)$.
			      \item This is a composition of continuous functions ($\arctan$ and a rational function whose denominator is non-zero away from the origin).
			      \item Thus, $f(x, y)$ is continuous for all $(x, y) \neq (0, 0)$.
		      \end{itemize}
		\item \textbf{Check continuity at the critical point $(0, 0)$:}
		      \begin{itemize}
			      \item \textbf{Condition for continuity:} We need $\lim_{(x,y) \to (0,0)} f(x, y) = f(0, 0)$.
			      \item We are given $f(0, 0) = \frac{\pi}{2}$.
			      \item We need to evaluate $\lim_{(x,y) \to (0,0)} \arctan\left(\frac{x}{x^2 + y^2}\right)$.
		      \end{itemize}
		\item \textbf{Strategy: Show the limit does not exist by restricting to a path.}
		      \begin{itemize}
			      \item If the limit exists, it must be the same regardless of the path taken towards $(0,0)$.
			      \item If we find even one path along which the limit doesn't exist, or two paths with different limits, then the overall limit does not exist.
			      \item \textit{Insight:} "Sometimes a single curve is enough to show the limit doesn't exist."
			      \item Let's analyze the argument of $\arctan$: $g(x, y) = \frac{x}{x^2 + y^2}$.
			      \item \textbf{Try the path $y=x$:} Substitute $y=x$ into $g(x,y)$ for $x \neq 0$:
			            \[ g(x, x) = \frac{x}{x^2 + x^2} = \frac{x}{2x^2} = \frac{1}{2x} \]
			      \item Compute the limit along this path as $x \to 0$:
			            \[ \lim_{x \to 0} g(x, x) = \lim_{x \to 0} \frac{1}{2x} \]
			      \item This limit does not exist (it tends to $+\infty$ as $x \to 0^+$ and $-\infty$ as $x \to 0^-$).
		      \end{itemize}
		\item \textbf{Conclusion for the limit of $f(x,y)$:}
		      \begin{itemize}
			      \item Since the limit of the argument $g(x,y) = \frac{x}{x^2+y^2}$ does not exist as $(x,y) \to (0,0)$ along the path $y=x$, the overall limit $\lim_{(x,y) \to (0,0)} g(x, y)$ does not exist.
			      \item Consequently, the limit $\lim_{(x,y) \to (0,0)} \arctan\left(g(x, y)\right)$ also does not exist.
			            (As $g \to \pm\infty$, $\arctan(g) \to \pm\frac{\pi}{2}$, which are different values depending on the direction along the path $y=x$, confirming non-existence).
		      \end{itemize}
		\item \textbf{Conclusion for continuity at $(0, 0)$:}
		      \begin{itemize}
			      \item Since $\lim_{(x,y) \to (0,0)} f(x, y)$ does not exist, the function cannot be continuous at $(0, 0)$. The condition $\lim_{(x,y) \to (0,0)} f(x, y) = f(0, 0)$ is not met.
		      \end{itemize}
		\item \textbf{Overall Conclusion:} The function $f(x, y)$ is continuous on $\mathbb{R}^2 \setminus \{(0, 0)\}$ but is discontinuous at the origin $(0, 0)$.
	\end{itemize}
\end{cascade}

\hfill

\begin{cascade}[Limit Existence and Continuity in $\mathbb{R}^2$ vs. Restricted Domain]
	\textbf{Key Questions:}
	\begin{enumerate}[a)]
		\item Does $\lim_{(x,y) \to (0,0)} f(x, y)$ exist?
		\item Is $f$ continuous at $(0,0)$?
		\item Is $f$ continuous in $D = \{(x,y) \in \mathbb{R}^2 : |y| \le x \le 1\}$?
	\end{enumerate}
	\textbf{Function Example:}
	\[ f(x, y) = \begin{cases} \frac{y^2}{x} & \text{if } x \neq 0 \\ 0 & \text{if } x = 0 \end{cases} \]
	\begin{itemize}
		\item \textbf{a) Existence of the limit at (0,0) in $\mathbb{R}^2$:}
		      \begin{itemize}
			      \item \textbf{Strategy:} Check limits along different paths approaching $(0,0)$. If they differ, the limit does not exist.
			      \item \textbf{Path 1: $y = \sqrt{x}$ (approaching from $x > 0$):}
			            \[ f(x, \sqrt{x}) = \frac{(\sqrt{x})^2}{x} = \frac{x}{x} = 1 \quad (\text{for } x > 0) \]
			            \[ \lim_{x \to 0^+} f(x, \sqrt{x}) = 1 \]
			      \item \textbf{Path 2: $y = x$ (approaching from $x \neq 0$):}
			            \[ f(x, x) = \frac{x^2}{x} = x \quad (\text{for } x \neq 0) \]
			            \[ \lim_{x \to 0} f(x, x) = \lim_{x \to 0} x = 0 \]
			      \item \textbf{Conclusion (a):} Since the limits along different paths ($1$ and $0$) are not equal, the limit $\lim_{(x,y) \to (0,0)} f(x, y)$ does not exist (when considered in $\mathbb{R}^2$).
		      \end{itemize}
		\item \textbf{b) Continuity at (0,0) in $\mathbb{R}^2$:}
		      \begin{itemize}
			      \item For continuity at $(0,0)$, the limit $\lim_{(x,y) \to (0,0)} f(x, y)$ must exist and equal $f(0,0)$.
			      \item From part (a), the limit does not exist.
			      \item \textbf{Conclusion (b):} Therefore, $f$ is not continuous at $(0,0)$ when considered as a function on $\mathbb{R}^2$.
		      \end{itemize}
		\item \textbf{c) Continuity in the domain $D = \{(x,y) \in \mathbb{R}^2 : |y| \le x \le 1\}$:}
		      \begin{itemize}
			      \item \textbf{Visualize D:} This domain is a region bounded by the lines $y=x$, $y=-x$, and $x=1$. It lies entirely in the $x \ge 0$ half-plane and includes the origin.
			      \item \textbf{Continuity for $x > 0$ in D:} In this region, $0 < x \le 1$. The function is $f(x,y) = y^2/x$. This is a ratio of polynomials with a non-zero denominator, hence continuous.
			      \item \textbf{Continuity at (0,0) *within D*:} We need to evaluate the limit as $(x,y) \to (0,0)$ such that $(x,y)$ remains in $D$. This means we approach $(0,0)$ satisfying $|y| \le x$.
			      \item \textbf{Using Squeeze Theorem within D:} For $(x,y) \in D$ and $x \neq 0$:
			            \begin{itemize}
				            \item Since $|y| \le x$, we have $y^2 \le x^2$ (as $x \ge 0$).
				            \item Since $x > 0$ in $D \setminus \{(0,0)\}$, we can divide by $x$: $\frac{y^2}{x} \le \frac{x^2}{x} = x$.
				            \item Also, $f(x,y) = \frac{y^2}{x} \ge 0$.
				            \item So, for $(x,y) \in D \setminus \{(0,0)\}$, we have $0 \le f(x,y) \le x$.
			            \end{itemize}
			      \item Take the limit as $(x,y) \to (0,0)$ *within D*. As $(x,y) \to (0,0)$ in D, we must have $x \to 0^+$.
			            \[ \lim_{\substack{(x,y) \to (0,0)}} 0 \le \lim_{\substack{(x,y) \to (0,0)}} f(x,y) \le \lim_{\substack{(x,y) \to (0,0)}} x \]
			            \[ 0 \le \lim_{\substack{(x,y) \to (0,0)}} f(x,y) \le 0 \]
			      \item By the Squeeze Theorem, $\lim_{\substack{(x,y) \to (0,0)}} f(x, y) = 0$.
			      \item This limit value (0) matches the function definition at the origin: $f(0,0) = 0$.
			      \item \textbf{Conclusion (c):} The function $f$ is continuous at every point $(x,y)$ with $x>0$ in $D$. The limit exists at $(0,0)$ *when restricted to D* and equals $f(0,0)$. Therefore, $f$ is continuous throughout the domain $D$.
		      \end{itemize}
	\end{itemize}
\end{cascade}

\hfill

\begin{cascade}[Proving Limit Non-Existence \& Pitfalls of Inequalities]
	\textbf{Key Question:} How to prove a limit does not exist at (0,0) and why might standard inequality methods fail?
	\textbf{Function Example:}
	\[ f(x, y) = \begin{cases} \frac{x^2 y}{x^4 + y^2} & \text{if } (x, y) \neq (0, 0) \\ 0 & \text{if } (x, y) = (0, 0) \end{cases} \]
	\textbf{Domain:} $\mathbb{R}^2$.
	\begin{itemize}
		\item \textbf{Continuity away from (0,0):}
		      \begin{itemize}
			      \item For $(x, y) \neq (0, 0)$, $f(x, y)$ is a ratio of polynomials. The denominator $x^4 + y^2 = 0$ only if $x=0$ and $y=0$.
			      \item Thus, $f(x, y)$ is continuous on $\mathbb{R}^2 \setminus \{(0, 0)\}$.
		      \end{itemize}
		\item \textbf{Investigating the limit at (0,0):}
		      \begin{itemize}
			      \item \textbf{Conjecture:} The limit $\lim_{(x,y) \to (0,0)} f(x, y)$ does not exist.
			      \item \textbf{Strategy: Path Restriction.} Find paths approaching (0,0) along which the function has different limits.
			      \item \textbf{Path 1: $y = x^2$ (Parabolic path):} Substitute $y=x^2$ into $f(x,y)$ for $x \neq 0$:
			            \[ f(x, x^2) = \frac{x^2 (x^2)}{x^4 + (x^2)^2} = \frac{x^4}{x^4 + x^4} = \frac{x^4}{2x^4} = \frac{1}{2} \]
			            \[ \lim_{x \to 0} f(x, x^2) = \frac{1}{2} \]
			      \item \textbf{Path 2: $y = 0$ (Along the x-axis):} Substitute $y=0$ into $f(x,y)$ for $x \neq 0$:
			            \[ f(x, 0) = \frac{x^2 (0)}{x^4 + 0^2} = \frac{0}{x^4} = 0 \]
			            \[ \lim_{x \to 0} f(x, 0) = 0 \]
			      \item \textbf{Conclusion on Limit:} Since the limits along different paths ($1/2$ and $0$) are not equal, the limit $\lim_{(x,y) \to (0,0)} f(x, y)$ does not exist.
		      \end{itemize}
		\item \textbf{Continuity at (0,0):}
		      \begin{itemize}
			      \item Since the limit does not exist, the function is not continuous at $(0,0)$. (The limit value would need to exist and be equal to $f(0,0)=0$).
		      \end{itemize}
		\item \textbf{Observation: Why Squeeze Theorem / Inequalities Can Be Misleading Here}
		      \begin{itemize}
			      \item \textbf{Attempt using AM-GM-like inequality:} We know $2ab \le a^2+b^2$. Let $a=x^2$ and $b=y$. Then $2x^2|y| \le (x^2)^2 + y^2 = x^4 + y^2$.
			      \item This gives $|f(x,y)| = \frac{x^2 |y|}{x^4 + y^2} \le \frac{\frac{1}{2}(x^4 + y^2)}{x^4 + y^2} = \frac{1}{2}$.
			      \item So, $0 \le |f(x,y)| \le \frac{1}{2}$. This shows the function is bounded near the origin.
			      \item \textbf{Why it fails:} This bound (1/2) does not tend to 0. Therefore, the Squeeze Theorem cannot be used to prove the limit is 0. It also doesn't prove the limit *isn't* 0, it's simply inconclusive for determining the limit value or existence based on squeezing towards 0.
			      \item \textbf{Another perspective:} Trying to bound $\frac{x^2}{x^4+y^2}$ by something that goes to zero fails because the $x^4$ term dominates the $x^2$ term along certain paths (like $y=0$), but not along others (like $y=x^2$ where $x^4+y^2 = 2x^4$, making the fraction $x^2/(2x^4) = 1/(2x^2)$ which diverges). The different powers ($x^2$ vs $x^4$) prevent a simple bounding function that works for all paths and tends to zero.
			      \item \textbf{Key Takeaway:} Path restriction is often necessary when the degrees of terms in the numerator and denominator are 'unbalanced' in a way that depends on the relationship between $x$ and $y$ (e.g., $y \approx x^2$). Simple inequalities might hide this path-dependent behavior.
		      \end{itemize}
	\end{itemize}
\end{cascade}

\clearpage

\subsection{Exercises on continuity, derivability and differentiability}

\begin{cascade}[Analyzing Continuity Derivability and Differentiability]
	In this class of exercises we have to check continuity, derivability and differentiability of a function in a given point or on its domain. \\

	\textbf{Function:} $f(x,y) = \begin{cases} \frac{ |x|y^4}{x^4+y^4} & \text{if } (x,y) \neq (0,0) \\ 0 & \text{if } (x,y) = (0,0) \end{cases}$ \\

	\begin{itemize}
		\item \textbf{1. Check Continuity at (0,0)}
		      \begin{itemize}
			      \item Compute the limit: $\lim_{(x,y)\to(0,0)} f(x,y)$.
			      \item Use Squeeze Theorem: $0 \leq \left| \frac{|x|y^4}{x^4+y^4} \right| = \frac{|x|y^4}{x^4+y^4}$.
			      \item Since $y^4 \leq x^4+y^4$, we have $\frac{y^4}{x^4+y^4} \leq 1$.
			      \item Thus, $0 \leq \frac{|x|y^4}{x^4+y^4} \leq |x|$.
			      \item As $(x,y) \to (0,0)$, we know $|x| \to 0$.
			      \item By the Squeeze Theorem, $\lim_{(x,y)\to(0,0)} f(x,y) = 0$.
			      \item Since the limit equals the function value $f(0,0)=0$, the function is \textbf{continuous} at (0,0).
		      \end{itemize}
		\item \textbf{2. Check Partial Derivatives at (0,0)}
		      \begin{itemize}
			      \item Use the limit definition for partial derivatives:
			      \item $\frac{\partial f}{\partial x}(0,0) = \lim_{h\to 0} \frac{f(h,0) - f(0,0)}{h} = \lim_{h\to 0} \frac{0-0}{h} = 0$.
			      \item $\frac{\partial f}{\partial y}(0,0) = \lim_{k\to 0} \frac{f(0,k) - f(0,0)}{k} = \lim_{k\to 0} \frac{0-0}{k} = 0$.
			      \item Both partial derivatives exist at (0,0). Thus, the function is \textbf{derivable} at (0,0) and the gradient is $\nabla f(0,0) = (0,0)$.
		      \end{itemize}
		\item \textbf{3. Check Directional Derivative at (0,0) for $w=(-1,-1)$}
		      \begin{itemize}
			      \item Normalize the direction vector $w$: $||w|| = \sqrt{(-1)^2+(-1)^2} = \sqrt{2}$.
			      \item The unit vector is $v = \frac{w}{||w||} = \left(-\frac{1}{\sqrt{2}}, -\frac{1}{\sqrt{2}}\right)$.
			      \item Compute $D_v f(0,0)$ using the limit definition: $D_v f(0,0) = \lim_{t\to 0} \frac{f(tv) - f(0,0)}{t}$.
			      \item \begin{align*}
				            D_v f(0,0) = \lim_{t\to 0} \frac{f(-\frac{t}{\sqrt{2}}, -\frac{t}{\sqrt{2}})}{t} = \lim_{t\to 0} \frac{1}{t} \left[ \frac{|-\frac{t}{\sqrt{2}}|(-\frac{t}{\sqrt{2}})^4}{(-\frac{t}{\sqrt{2}})^4 + (-\frac{t}{\sqrt{2}})^4} \right] \\
				            = \lim_{t\to 0} \frac{1}{t} \left[ \frac{\frac{|t|}{\sqrt{2}} \frac{t^4}{4}}{\frac{t^4}{4} + \frac{t^4}{4}} \right] = \lim_{t\to 0} \frac{1}{t} \left[ \frac{\frac{|t|t^4}{4\sqrt{2}}}{\frac{2t^4}{4}} \right] = \lim_{t\to 0} \frac{|t|}{2\sqrt{2} t}
			            \end{align*}
			      \item This limit \textbf{does not exist} because the limits from the left ($t \to 0^-$) and right ($t \to 0^+$) are different ($-\frac{1}{2\sqrt{2}}$ and $\frac{1}{2\sqrt{2}}$, respectively).
		      \end{itemize}
		\item \textbf{4. Check Differentiability at (0,0)}
		      \begin{itemize}
			      \item \textbf{Method 1: Using Directional Derivative Result}
			            \begin{itemize}
				            \item A necessary condition for differentiability at a point is the existence of all directional derivatives at that point.
				            \item Since $D_v f(0,0)$ for $v=(-\frac{1}{\sqrt{2}}, -\frac{1}{\sqrt{2}})$ does not exist, $f$ is \textbf{not differentiable} at (0,0).
			            \end{itemize}
			      \item \textbf{Method 2: Using the Definition of Differentiability}
			            \begin{itemize}
				            \item Check if the limit defining differentiability is zero:
				            \item \begin{align*}
					                  \lim_{(h,k)\to(0,0)} \frac{f(h,k) - f(0,0) - \nabla f(0,0) \cdot (h,k)}{\sqrt{h^2+k^2}} \\
					                  \lim_{(h,k)\to(0,0)} \frac{f(h,k) - 0 - (0,0) \cdot (h,k)}{\sqrt{h^2+k^2}}              \\
					                  = \lim_{(h,k)\to(0,0)} \frac{\frac{|h|k^4}{h^4+k^4}}{\sqrt{h^2+k^2}} = \lim_{(h,k)\to(0,0)} \frac{|h|k^4}{(h^4+k^4)\sqrt{h^2+k^2}}
				                  \end{align*}
				            \item Test the limit along the path $h=k$:
				            \item $\lim_{h\to 0} \frac{|h|h^4}{(h^4+h^4)\sqrt{h^2+h^2}} = \lim_{h\to 0} \frac{|h|h^4}{2h^4\sqrt{2h^2}} = \lim_{h\to 0} \frac{|h|h^4}{2h^4\sqrt{2}|h|} = \frac{1}{2\sqrt{2}}$.
				            \item Since the limit must be 0 for differentiability, and we found a path where the limit is $\frac{1}{2\sqrt{2}} \neq 0$, the function $f$ is \textbf{not differentiable} at (0,0).
			            \end{itemize}
		      \end{itemize}
	\end{itemize}
\end{cascade}

\hfill

\begin{cascade}[Analysis a $\arctan$ function]
	\textbf{Function:} $f(x,y) = x \arctan\left(\frac{y}{x}\right)$
	\begin{itemize}
		\item \textbf{1. Determine the Domain}
		      \begin{itemize}
			      \item The argument of $\arctan$ requires the denominator $x$ to be non-zero.
			      \item Domain $D = \{(x,y) \in \mathbb{R}^2 \mid x \neq 0\}$.
			      \item This domain represents the entire $\mathbb{R}^2$ plane excluding the y-axis.
			      \item The domain is open, not connected, and not bounded.
		      \end{itemize}
		\item \textbf{2. Determine where $f(x,y) \ge 0$}
		      \begin{itemize}
			      \item We need $x \arctan(\frac{y}{x}) \ge 0$.
			      \item Case 1: $x > 0$. Requires $\arctan(\frac{y}{x}) \ge 0$, which implies $\frac{y}{x} \ge 0$. Since $x>0$, this means $y \ge 0$.
			      \item Case 2: $x < 0$. Requires $\arctan(\frac{y}{x}) \le 0$, which implies $\frac{y}{x} \le 0$. Since $x<0$, this means $y \ge 0$.
			      \item Combining both cases, $f(x,y) \ge 0$ if and only if $y \ge 0$ (and $x \neq 0$).
			      \item The set is $\{(x,y) \in \mathbb{R}^2 \mid x \neq 0, y \ge 0\}$. (The upper half-plane excluding the y-axis).
		      \end{itemize}
		\item \textbf{3. Continuous Extension to $\mathbb{R}^2$}
		      \begin{itemize}
			      \item The function is continuous on its domain $D$ as it's composed of continuous functions.
			      \item To extend it continuously to $\mathbb{R}^2$, we need to define it for $x=0$ such that the limit matches the defined value.
			      \item Consider the limit as $(x,y) \to (0, y_0)$ for any $y_0 \in \mathbb{R}$.
			      \item $\lim_{(x,y)\to(0, y_0)} x \arctan\left(\frac{y}{x}\right)$.
			      \item Since $x \to 0$ and $\arctan(\frac{y}{x})$ is bounded (between $-\pi/2$ and $\pi/2$), the limit is $0$.
			      \item We can define a continuous extension $\tilde{f}: \mathbb{R}^2 \to \mathbb{R}$ as:
			            $$ \tilde{f}(x,y) = \begin{cases} x \arctan(\frac{y}{x}) & \text{if } x \neq 0 \\ 0 & \text{if } x = 0 \end{cases} $$
			      \item This function $\tilde{f}$ is continuous on all of $\mathbb{R}^2$.
		      \end{itemize}
		\item \textbf{4. Gradient and Directional Derivatives at (1,0)}
		      \begin{itemize}
			      \item Compute partial derivatives for $x \neq 0$:
			            \begin{itemize}
				            \item $\frac{\partial f}{\partial x} = \arctan\left(\frac{y}{x}\right) + x \cdot \frac{1}{1+(y/x)^2} \cdot \left(-\frac{y}{x^2}\right) = \arctan\left(\frac{y}{x}\right) - \frac{xy}{x^2+y^2}$.
				            \item $\frac{\partial f}{\partial y} = x \cdot \frac{1}{1+(y/x)^2} \cdot \left(\frac{1}{x}\right) = \frac{x^2}{x^2+y^2}$.
			            \end{itemize}
			      \item Evaluate at the point $(1,0)$:
			            \begin{itemize}
				            \item $\frac{\partial f}{\partial x}(1,0) = \arctan(0) - \frac{1 \cdot 0}{1^2+0^2} = 0$.
				            \item $\frac{\partial f}{\partial y}(1,0) = \frac{1^2}{1^2+0^2} = 1$.
			            \end{itemize}
			      \item The gradient is $\nabla f(1,0) = (0, 1)$.
			      \item Since the partial derivatives are continuous around $(1,0)$ (as $x=1 \neq 0$), the function $f$ is differentiable at $(1,0)$.
			      \item \textbf{Conclusion on Directional Derivatives:} Because $f$ is differentiable at $(1,0)$, all directional derivatives $D_v f(1,0)$ exist and can be calculated using the formula $D_v f(1,0) = \nabla f(1,0) \cdot v$.
		      \end{itemize}
	\end{itemize}
\end{cascade}

\hfill

\begin{cascade}[Gradient and Tangent Line to Contour Line]
	\textbf{Function:} $f(x,y) = \frac{\sqrt{xy}}{e^{2x}}$ \quad (Note: Domain requires $xy \ge 0$)

	\textbf{Point:} $P_0 = (-1, -1)$ (Point is in the domain as $(-1)(-1) = 1 \ge 0$)
	\begin{itemize}
		\item \textbf{1. Compute the Gradient $\nabla f(-1,-1)$}
		      \begin{itemize}
			      \item Compute partial derivative with respect to $x$:
			            $$ \frac{\partial f}{\partial x} = \frac{(\frac{1}{2\sqrt{xy}} \cdot y) e^{2x} - \sqrt{xy} (2e^{2x})}{(e^{2x})^2} = \frac{e^{2x}(\frac{y}{2\sqrt{xy}} - 2\sqrt{xy})}{e^{4x}} = \frac{y - 4xy}{2\sqrt{xy} e^{2x}} $$
			      \item Evaluate $\frac{\partial f}{\partial x}$ at $(-1,-1)$:
			            $$ \frac{\partial f}{\partial x}(-1,-1) = \frac{-1 - 4(-1)(-1)}{2\sqrt{(-1)(-1)} e^{2(-1)}} = \frac{-1 - 4}{2\sqrt{1} e^{-2}} = \frac{-5}{2e^{-2}} = -\frac{5}{2}e^2 $$
			      \item Compute partial derivative with respect to $y$:
			            $$ \frac{\partial f}{\partial y} = \frac{1}{e^{2x}} \cdot \frac{\partial}{\partial y}(\sqrt{xy}) = \frac{1}{e^{2x}} \cdot \frac{1}{2\sqrt{xy}} \cdot x = \frac{x}{2\sqrt{xy} e^{2x}} $$
			      \item Evaluate $\frac{\partial f}{\partial y}$ at $(-1,-1)$:
			            $$ \frac{\partial f}{\partial y}(-1,-1) = \frac{-1}{2\sqrt{(-1)(-1)} e^{2(-1)}} = \frac{-1}{2\sqrt{1} e^{-2}} = \frac{-1}{2e^{-2}} = -\frac{1}{2}e^2 $$
			      \item The Gradient at $(-1,-1)$ is:
			            $$ \nabla f(-1,-1) = \left( -\frac{5}{2}e^2, -\frac{1}{2}e^2 \right) $$
		      \end{itemize}
		\item \textbf{2. Find Tangent Line to contour line at $(-1,-1)$}
		      \begin{itemize}
			      \item \textbf{Principle:} The gradient vector $\nabla f(x_0, y_0)$ is orthogonal to the contour line of $f$ passing through $(x_0, y_0)$.
			      \item The tangent line at $(x_0, y_0)$ must be orthogonal to $\nabla f(x_0, y_0)$.
			      \item Our point is $P_0 = (-1, -1)$ and gradient is $\nabla f(-1,-1) = (-\frac{5}{2}e^2, -\frac{1}{2}e^2)$.
			      \item Find a direction vector $w = (w_1, w_2)$ for the tangent line such that $w \cdot \nabla f(-1,-1) = 0$.
			            $$ w_1 \left(-\frac{5}{2}e^2\right) + w_2 \left(-\frac{1}{2}e^2\right) = 0 $$
			      \item Divide by $-\frac{1}{2}e^2$: $5w_1 + w_2 = 0$.
			      \item Choose a simple direction vector satisfying this, e.g., $w_1 = 1$, $w_2 = -5$. So, $w = (1, -5)$.
			      \item The tangent line passes through $P_0 = (-1, -1)$ with direction vector $w = (1, -5)$.
			      \item \textbf{Parametric Equation:} $r(t) = P_0 + tw = (-1, -1) + t(1, -5)$
			            $$ \begin{cases} x = -1 + t \\ y = -1 - 5t \end{cases} $$
			      \item \textbf{Cartesian Equation:} Eliminate $t$. From the first equation, $t = x + 1$.
			            $$ y = -1 - 5(x+1) = -1 - 5x - 5 $$
			            $$ y = -5x - 6 $$
			      \item The equation of the tangent line to the contour line at $(-1,-1)$ is $y = -5x - 6$.
		      \end{itemize}
	\end{itemize}
\end{cascade}

\clearpage
